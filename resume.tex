%-------------------------
% Resume in Latex
% Author : Jake Gutierrez
% Based off of: https://github.com/sb2nov/resume
% License : MIT
%------------------------

\documentclass[letterpaper,11pt]{article}

\usepackage{latexsym}
\usepackage[empty]{fullpage}
\usepackage{titlesec}
\usepackage{marvosym}
\usepackage[usenames,dvipsnames]{color}
\usepackage{verbatim}
\usepackage{enumitem}
\usepackage[hidelinks]{hyperref}
\usepackage{fancyhdr}
\usepackage[english]{babel}
\usepackage{tabularx}
\input{glyphtounicode}


%----------FONT OPTIONS----------
% sans-serif
% \usepackage[sfdefault]{FiraSans}
% \usepackage[sfdefault]{roboto}
% \usepackage[sfdefault]{noto-sans}
% \usepackage[default]{sourcesanspro}

% serif
% \usepackage{CormorantGaramond}
% \usepackage{charter}


\pagestyle{fancy}
\fancyhf{} % clear all header and footer fields
\fancyfoot{}
\renewcommand{\headrulewidth}{0pt}
\renewcommand{\footrulewidth}{0pt}

% Adjust margins
\addtolength{\oddsidemargin}{-0.5in}
\addtolength{\evensidemargin}{-0.5in}
\addtolength{\textwidth}{1in}
\addtolength{\topmargin}{-.5in}
\addtolength{\textheight}{1.0in}

\urlstyle{same}

\raggedbottom
\raggedright
\setlength{\tabcolsep}{0in}

% Sections formatting
\titleformat{\section}{
  \vspace{-4pt}\scshape\raggedright\large
}{}{0em}{}[\color{black}\titlerule \vspace{-5pt}]

% Ensure that generate pdf is machine readable/ATS parsable
\pdfgentounicode=1

%-------------------------
% Custom commands
\newcommand{\resumeItem}[1]{
  \item\small{
    {#1 \vspace{-2pt}}
  }
}

\newcommand{\resumeSubheading}[4]{
  \vspace{-2pt}\item
    \begin{tabular*}{0.97\textwidth}[t]{l@{\extracolsep{\fill}}r}
      \textbf{#1} & #2 \\
      \textit{\small#3} & \textit{\small #4} \\
    \end{tabular*}\vspace{-7pt}
}

\newcommand{\resumeSubSubheading}[2]{
    \item
    \begin{tabular*}{0.97\textwidth}{l@{\extracolsep{\fill}}r}
      \textit{\small#1} & \textit{\small #2} \\
    \end{tabular*}\vspace{-7pt}
}

\newcommand{\resumeProjectHeading}[2]{
    \item
    \begin{tabular*}{0.97\textwidth}{l@{\extracolsep{\fill}}r}
      \small#1 & #2 \\
    \end{tabular*}\vspace{-7pt}
}

\newcommand{\resumeSubItem}[1]{\resumeItem{#1}\vspace{-4pt}}

\renewcommand\labelitemii{$\vcenter{\hbox{\tiny$\bullet$}}$}

\newcommand{\resumeSubHeadingListStart}{\begin{itemize}[leftmargin=0.15in, label={}]}
\newcommand{\resumeSubHeadingListEnd}{\end{itemize}}
\newcommand{\resumeItemListStart}{\begin{itemize}}
\newcommand{\resumeItemListEnd}{\end{itemize}\vspace{-5pt}}

%-------------------------------------------
%%%%%%  RESUME STARTS HERE  %%%%%%%%%%%%%%%%%%%%%%%%%%%%


\begin{document}

%----------HEADING----------

% \begin{tabular*}{\textwidth}{l@{\extracolsep{\fill}}r}
%   \textbf{\href{http://sourabhbajaj.com/}{\Large Sourabh Bajaj}} & Email : \href{mailto:sourabh@sourabhbajaj.com}{sourabh@sourabhbajaj.com}\\
%   \href{http://sourabhbajaj.com/}{http://www.sourabhbajaj.com} & Mobile : +1-123-456-7890 \\
% \end{tabular*}

\begin{center}
    \textbf{\Huge \scshape Asif Al Noor} \\ \vspace{1pt}
     \small RF Engineer, Vancouver BC \\
    \small 250-859-0124 $|$ \href{mailto:asif.alnoor@alumni.ubc.ca}{\underline{asif.alnoor@alumni.ubc.ca}} $|$ 
    \href{https://roonlafisa.github.io/}{\underline{roonlafisa.github.io}}
\end{center}

%-----------SUMMARY-----------
\section{Summary}

\resumeSubHeadingListStart
    \resumeProjectHeading {}{}
    Intermediate radio frequency engineer with a strong background in electromagnetics and more than 5 years of experience in developing and testing RF components and devices from concept through to manufacture.
        \resumeItemListStart
        \resumeItem{In-depth knowledge of radio systems and components design, development, verification and optimization.}
        \resumeItem{Strong capabilities of antenna research, design simulation, fabrication and testing using industry-standard modeling tools: CST, ADS, HFSS, Feko.}
        \resumeItem{Experience in RF field measurements, as well as characterization and evaluation of RF products and systems.}
        \resumeItem{Strong programming and scripting skills: Python, MATLAB, C/C++.}
        \resumeItem{Extensive experience with device debugging and QC process, and lab equipment: VNA, signal generators, spectrum analyzer, oscilloscopes.}
        \resumeItem{Solid knowledge of Signals and Systems, Electromagnetics and Transmission Line Theory, Digital Communications Theory, Wireless Systems and Radio Frequency/Microwave Circuits.}
        \resumeItemListEnd
\resumeSubHeadingListEnd
%-----------EDUCATION-----------
\section{Education}
  \resumeSubHeadingListStart
    \resumeSubheading
      {University of British Columbia}{Kelowna, BC}
      {MASc in Electrical Engineering. Authored 2 publications. \href{https://dx.doi.org/10.14288/1.0319322}{\underline{thesis URL}}}{Aug. 2014 -- Oct. 2016}
    \resumeSubheading
      {Islamic University of Technology}{Dhaka, Bangladesh}
      {BSc in Electrical and Electronic Engineering. Co-authored 5 publications.}{Jan. 2010 -- Oct. 2013}
  \resumeSubHeadingListEnd


%-----------EXPERIENCE-----------
\section{Professional Experience}
  \resumeSubHeadingListStart

    \resumeSubheading
      {SMT Research Ltd.}{April 2020 -- Present}
      {Intermediate RF Engineer}{Vancouver, BC}
      \resumeItemListStart
   %     \resumeItem{Designing new products/solutions, and enhancing existing data-loggers and sensors providing wireless data about structural monitoring technology. }
        \resumeItem{Modernized, developed, tested and installed wireless system upgrade to the existing structural monitoring system to reduce installation costs while maintaining computational accuracy.}
        \resumeItem{Designed, tested and verified antenna and other components for an RF power harvesting circuit at 915 MHz. The circuit performs measurements from a battery-free sensor network.}
         \resumeItem{Pioneered custom battery-assisted RFID passive sensor tags for remote datalogging and developed an RFID based tracking system to pinpoint hidden sensor location.}
        \resumeItem{Carried out numerical analysis, software simulation to characterize electrical properties and RF propagation through various roof assemblies. Developed test setup to run field experiments and verify the characterization. }
        \resumeItem{Performed QC tests on sensors and data-loggers, and investigated circuit hardware failures.}
        \resumeItem{Installed sensors and dataloggers in active sites and provided technical support to technicians during installation.}
        \resumeItem{Drafted and edited testing procedures, reports, papers and other documents.}
   %     \resumeItem{Inventor in one patent application (pending).}
      \resumeItemListEnd
      
% -----------Multiple Positions Heading-----------
%    \resumeSubSubheading
%     {Software Engineer I}{Oct 2014 - Sep 2016}
%     \resumeItemListStart
%        \resumeItem{Apache Beam}
%          {Apache Beam is a unified model for defining both batch and streaming data-parallel processing pipelines}
%     \resumeItemListEnd
%    \resumeSubHeadingListEnd
%-------------------------------------------

    \resumeSubheading
      {Direct Kinetic Solutions}{Oct. 2019 -- March 2021}
      {RF Engineering Consultant}{El Paso, TX (Remote)}
      \resumeItemListStart
   %     \resumeItem{Developed RF solutions for clients based on uninterrupted power sources technology.}
        \resumeItem{Developed RF system and cubesat EPS, researched solutions, recommended equipment  based on client tech.}
        \resumeItem{Produced preliminary cubesat payload concepts for clients, including antenna deployment, antenna performance, payload block diagram, link budgets, power budgets, mass budgets.}
        \resumeItem{Developed RF solutions for the US Army contract bids and wrote technical proposals within strict deadlines.}
        \resumeItem{Proposed a standard 6U CubeSat platform to obtain a high resolution (5 m colour and 3m monochromatic) image and high-definition (HD) movie for high-speed ISR applications.}
        \resumeItem{Identified new product opportunities, market trends and competitiveness in the marketplace.}
    \resumeItemListEnd

    \resumeSubheading
      {Helios Wire Inc.}{July 2017 -- May 2019}
      {Lead RF Engineer}{Vancouver, BC}
      \resumeItemListStart
   %     \resumeItem{Spearheaded the RF system design towards developing contemporary wireless communication between nano-satellite constellation and ground segment in a startup environment.}
        \resumeItem{Designed, simulated and successfully deployed CubeSat C-, X- and S-Band antennas for CubeSat data and TT\&C communication. Conducted studies with a 3 ft S-band reflector antenna in the ground station.}
        \resumeItem{Performed system analysis and carried out calculations such as switch/hybrid and filter assemblies, link-budget, power budget, power flux density, mass budget analysis, etc.}
        \resumeItem{Conceptualized and modelled an outdoor-rated access point/gateway to connect to Cubesat and IoT tags.}
        \resumeItem{Committed to the new product development process, such as defining product requirement documents (PRD), developing product roadmap, and producing the conceptual design.}
   %     \resumeItem{Carried out basic simulations of orbit to ground CubeSat communications via STK.}
        \resumeItem{Interfaced with subcontractors to determine RF product requirements, negotiate prices and terms.}
        \resumeItem{Collaborated with vendors to troubleshoot antenna production as well as to purchase equipment, including Power Amplifiers, LNAs, Filters, etc., for RF subsystem.}
      \resumeItemListEnd

  \resumeSubHeadingListEnd

%-----------ACADEMIC EXPERIENCE-----------
\section{Academic Experience}
  \resumeSubHeadingListStart

    \resumeSubheading
      {Markley Electromagnetics Research Group, UBC}{Sept. 2014 - Dec. 2016}
      {Research Assistant}{Kelowna, BC}
      \resumeItemListStart
        \resumeItem{Investigated passive electromagnetic architectures and solutions, such as optical sensors, metamaterials, antennas, frequency selective surfaces, and wireless power transfer. }
        \resumeItem{Developed and characterized a planar broadband leaky-wave antenna for planar applications using COMSOL.}
      \resumeItemListEnd
   
    \resumeSubSubheading
     {Teaching Assistant}{}
     \resumeItemListStart
        \resumeItem{Taught APSC 178 (Electricity, Magnetism, \& Waves) course for two semesters, held office hours and received excellent reviews from the first year engineering students.}
     \resumeItemListEnd
     
    \resumeSubheading
      {Electromagnetics Research Group, IUT}{Nov. 2012 - Dec. 2013}
      {Undergraduate Research Assistant}{Dhaka, Bangladesh}
      \resumeItemListStart
        \resumeItem{Researched characteristics of Surface-Plasmon-Polariton (SPP) through various waveguides and determined the optimum design for different applications.}
      \resumeItemListEnd
      
    \resumeSubHeadingListEnd
    
    

%-----------SELECTED PROJECTS-----------
\section{Selected Projects}
    \resumeSubHeadingListStart
      \resumeProjectHeading
          {\textbf{Microwave Amplifier Design} $|$ \emph{AWR, VNA, Signal Generator}}{March 2015}
        \resumeItemListStart
         \resumeItem{Designed a UHF microwave amplifier at 1 GHz using AWR microwave office and adopted it on Rogers RO4350 substrate with NXP BFR520 transistor.}
        \resumeItemListEnd
        
      \resumeProjectHeading
          {\textbf{Fabrication of Graded-dielectric Materials for Antenna Applications} $|$ \emph{MATLAB, VNA}}{Oct 2018}
        \resumeItemListStart
         \resumeItem{Developed a MATLAB script to produce spatially varying permittivity by drilling holes on Rogers RT/duroid 5880,TMM 3,TMM 4 and TMM 6 high frequency laminates.}
        \resumeItemListEnd
    \resumeSubHeadingListEnd



%
%-----------PUBLICATIONS-----------
\section{Selected Publications}
 \begin{itemize}[leftmargin=0.15in, label={\tiny$\bullet$}]
    \small{
    \item{A Geometrically Phase-Compensated Transformation Optics Superstrate for Fixed-Beam Broadband Leaky-Wave Radiation, IEEE Explore, 2019 - \href{https://ieeexplore.ieee.org/document/8739640}{\textit{\underline{publication URL}}}}
    \item{ Achieving Linear Phase Through Geometrically-Compensated Transformation Domains for Leaky-Wave Antenna Radiation, IEEE Explore, 2016 - \href{https://ieeexplore.ieee.org/document/7695749}{\textit{\underline{publication URL}}}}
    }
 \end{itemize}


%-----------PROGRAMMING SKILLS-----------
\section{Technical Skills}
 \begin{itemize}[leftmargin=0.15in, label={}]
    \small{\item{
     \textbf{Softwares}{: ADS, HFSS, CST, KiCAD, STK} \\
     \textbf{Languages}{: MATLAB, GNU Octave, Python, C/C++, Simulink, R} \\
     \textbf{Equipment}{: VNA, Signal generator, Spectrum analyzer, Oscilloscope}
    }}
 \end{itemize}

%
%-----------PROFESSIONAL AFFILIATIONS-----------
\section{Professional Affiliations}
    \resumeSubHeadingListStart
      \resumeProjectHeading
          {\textbf{Engineers and Geoscientists BC} $|$ \emph{Engineer-in-Training (EIT) }}{2016 -- Present}

      \resumeProjectHeading
          {\textbf{IEEE Internet of Things Society} $|$ \emph{Member}}{2018 -- Present}

    \resumeSubHeadingListEnd
    
%-----------SELECTED PROJECTS-----------
\section{Extracurricular Experience}
  \resumeSubHeadingListStart

    \resumeSubheading
      {Zen Maker Lab}{Oct. 2019 - March. 2020}
      {STEM Educator}{Vancouver, BC}
      \resumeItemListStart
        \resumeItem{Designed aerospace and electronics STEM curriculum to provide kids hands-on experience real electronic components.}
        \resumeItem{Fabricated various hobby electric circuits to make 3D printed electronic toys for enthusiastic children.}
      \resumeItemListEnd

    \resumeSubHeadingListEnd
    
%-------------------------------------------
\end{document}
